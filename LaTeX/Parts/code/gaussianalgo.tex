\begin{lstlisting}[language=Python, style=jupycolors]
def gaussianSolver(expA, nbeq, nbsol):
    # forward steps
    for i in range (0,nbeq-1):   # range starts at 0 finish at nbeq-2 
        if expA[i,i] == 0:
            swipeResult = swapLines(expA, i, nbeq)
            if swipeResult[0] == 0: # no line have been swapped
                print("Infinite solutions")
            else:
                expA = swipeResult[1]
        for j in range (i+1,nbeq):  # range starts i+1 and finish at nbeq-1
            if abs(expA[i,i]) > 1e-8:
                expA[j,i:nbeq+nbsol]=expA[j,i:nbeq+nbsol]-expA[i,i:nbeq+nbsol]*expA[j,i]/expA[i,i];
                expA[j,i]=0
    
    #  backward steps:
    sols=np.zeros(shape=(nbeq,nbsol))
    for i in range (nbeq-1,-1,-1):    # range starts at nbeq-1 and finish 0 (third param is step)
        # we can have some values like 1e-15 that are not considered as 0 whereas it should
        if abs(expA[i, :nbeq-1].any()) < 1e-8 and abs(expA[i, nbeq]) < 1e-8:
            print ("infinite x" + str(i+1) + " solutions, taking 0")
        elif abs(expA[i, :nbeq].any()) < 1e-8 and abs(expA[i, nbeq]) > 1e-8:
            return "No solution for this system"
        else:
            sols[i,:]=(expA[i,nbeq:nbeq+nbsol]-expA[i,i+1:nbeq]*sols[i+1:nbeq,:])/expA[i,i]
    return sols
\end{lstlisting}