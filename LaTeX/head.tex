
\author{Valentin SERVIERES}
\pdfminorversion=7 % To use charte graphique (pdf 1.7)
\usepackage[utf8]{inputenc}
\usepackage[T1]{fontenc}
\usepackage{lscape}
\usepackage{subcaption, boldline,multirow,tabularx,colortbl,makecell,fancybox,amsfonts,amssymb,amsmath,mathrsfs,array, svg}
\usepackage{pgf,tikz,xcolor,graphicx}
\usetikzlibrary{calc,positioning,shapes.geometric,shapes.symbols,shapes.misc, fit, shapes, arrows, arrows.meta,fadings,through}
\usepackage[top=2cm, bottom=2cm, left=2cm, right=2cm]{geometry}
\usepackage{hyperref,titlesec,eurosym,eso-pic,float}

\usepackage[english]{babel}
%\usepackage{bibleref} ajouté par moi mais flemme
% minted et les encars de code en gros
%\usepackage[newfloat]{minted}
\usepackage{caption}
\usepackage{tcolorbox}
%\newenvironment{code}{\captionsetup{type=listing}}{}
%\SetupFloatingEnvironment{listing}{name=Code Source}
\tcbuselibrary{skins, breakable}
\newtcolorbox{resultbox}{
  enhanced,
  colframe=blue!20!black,
  colback=blue!10,
  arc=0mm,
  boxrule=0.8pt,
  left=1mm,
  right=1mm,
  top=1mm,
  bottom=1mm,
  breakable,
  pad at break=0mm,
}

% table des annexes
\usepackage{minitoc}
\usepackage{pdfpages}

\usepackage{color}
\definecolor{white}{rgb}{1,1,1}
\definecolor{gray}{rgb}{.5,.5,.5}
\definecolor{lightgray}{rgb}{.9,.9,.9}
\definecolor{darkgray}{rgb}{.4,.4,.4}
\definecolor{green}{rgb}{0,.3,0}
\definecolor{red}{rgb}{0.6,0,0}
\definecolor{orange}{rgb}{0.6,0.3,0}
\definecolor{blue}{rgb}{0,0,0.5}
\definecolor{teal}{rgb}{0,0.3,0.5}
\definecolor{mauve}{rgb}{0.8,0.3,0.7}
\definecolor{purple}{rgb}{0.65, 0.12, 0.82}
\definecolor{tealbox}{rgb}{0.5,0.7,1}
\definecolor{orangebox}{rgb}{1,0.7,0.5}


\definecolor{pyred}{rgb}{0.4,0,0}
\definecolor{pygreen}{rgb}{0,0.4,0}
\definecolor{pycyan}{rgb}{0,0.3,0.3}
\definecolor{pyblue}{rgb}{0,0,0.7}
\definecolor{pymagenta}{rgb}{0.5,0,0.4}
\definecolor{pylightgray}{rgb}{0.97,0.97,0.97}
\definecolor{pykey}{rgb}{0.117,0.403,0.713}

% bibliographie
%\usepackage{biblatex}
%\usepackage{csquotes}
%\addbibresource{bibliography.bib}

% for code embed but prefer lstlisting
%\usepackage{minted}
%\usepackage{listings}
\usepackage[procnames]{listings}

\DeclareMathOperator*{\Max}{max}